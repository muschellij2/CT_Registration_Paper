% latex table generated in R 3.1.0 by xtable 1.7-3 package
% Tue Jul 15 14:35:48 2014
\begin{table}[ht]
\centering
\begin{tabular}{lccc}
  \hline
Area & Population Prevalence & NIHSS HPR & GCS HPR \\ 
  \hline
CSF & 7.9 & 10.0 & 4.2 \\ 
  Insular & 4.7 &  &  \\ 
  Superior temporal gyrus & 3.8 &  &  \\ 
  Putamen left & 3.0 &  &  \\ 
  Insular right & 2.9 &  &  \\ 
  External capsule left & 2.3 &  &  \\ 
  Superior corona radiata left & 1.9 & 11.8 & 27.9 \\ 
  Superior temporal wm left & 1.9 &  &  \\ 
  Superior corona radiata right & 1.8 &  &  \\ 
  Putamen right & 1.8 &  &  \\ 
  Posterior limb of internal capsule left &  & 10.1 & 3.9 \\ 
  Thalamus left &  & 7.6 & 33.9 \\ 
  Caudate nucleus left &  & 5.4 & 9.6 \\ 
  Superior longitudinal fasciculus left &  & 4.9 & 5.9 \\ 
  Globus pallidus left &  & 3.7 &  \\ 
  Anterior limb of internal capsule left &  & 3.6 &  \\ 
  Outside brain mask &  & 3.5 &  \\ 
  Anterior limb of internal capsule right &  & 3.0 &  \\ 
  Postcentral wm left &  &  & 6.7 \\ 
  Posterior corona radiata left &  &  & 3.1 \\ 
  Precentral wm left &  &  & 1.3 \\ 
  Supramarginal wm left &  &  & 1.1 \\ 
   \hline
\end{tabular}
\caption{Distribution of the top 10 areas of engagement for population  3D histogram, the NIHSS HPR based on a p-value  threshold of .01, the GCS HPR based on voxels with 1000 smallest p-values.  Each value represents the percentage of the HPR engaged in this area.  The population-level areas are percentages weighted by proportion. Each distribution  of areas is based on the Eve atlas.  We see that the population is engaged in areas of the CSF, such as the ventricles, and  the insular and putaminal regions. The HPR based on the NIHSS analysis engages primarily areas of the internal capsule and ventricular regions. The HPR based on the GCS analysis engages primarily the left thalamus and superior corona radiata. These percentages   can be calculated on a per-scan level as well.} 
\label{t:breakdown}
\end{table}
