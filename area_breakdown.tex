% latex table generated in R 3.0.2 by xtable 1.7-3 package
% Wed Jun 11 14:12:55 2014
\begin{table}[ht]
\centering
\begin{tabular}{lccc}
  \hline
Area & Population Engagement & NIHSS HPR & GCS HPR \\ 
  \hline
Globus Pallidus: Total & 20.3 & 40.0 & 0.0 \\ 
  Globus Pallidus: Right & 14.8 & 34.8 & 0.0 \\ 
  Globus Pallidus: Left & 25.2 & 44.7 & 0.0 \\ 
  Putamen: Total & 23.3 & 6.6 & 0.0 \\ 
  Putamen: Right & 17.5 & 3.8 & 0.0 \\ 
  Putamen: Left & 29.2 & 9.4 & 0.0 \\ 
  Thalamus: Total & 7.9 & 8.9 & 1.7 \\ 
  Thalamus: Right & 6.8 & 3.1 & 0.0 \\ 
  Thalamus: Left & 9.1 & 14.6 & 3.4 \\ 
   \hline
\end{tabular}
\caption{Engagement of the Thalamus, Putamen and Globus Pallidus by the population  3D histogram, the NIHSS HPR based on a p-value  threshold of $.01$, the GCS HPR based on voxels with $1000$ smallest p-values.  Each distribution  is based on the Eve atlas.  The population engagement represents the mean proportion of engagement for that brain region over all patient ICH masks.  The HPR columns represent the percent of voxels in that brain region that  are in the HPR for NIHSS and GCS scores.   On average, $23\%$ of the Putamen, $20\%$ of the Globus Pallidus, and $8\%$ of the Thalmus are  engaged with ICH from patients in this study.  The HPR from the NIHSS analysis engages $40\%$ of the Globus Pallidus, $6\%$ of the Putamen, and $9\%$ of the Thalamus.  The HPR from the GCS analysis engages only $2\%$ of the Thalamus, but not the Putamen nor the Globus Pallidus; the GCS HPR is only $1000$ voxels.  All engagement is higher on the left side compared to the right. The Eveatlas can be used to calculate area engagement on a per-scan level as well to give, for example, putaminal engagement with ICH at a scan level. } 
\label{t:area_breakdown}
\end{table}
