
% Table created by stargazer v.5.1 by Marek Hlavac, Harvard University. E-mail: hlavac at fas.harvard.edu
% Date and time: Tue, Jun 10, 2014 - 14:39:35
\begin{table}[!htbp] \centering 
\begin{tabular}{@{\extracolsep{5pt}}lcc|cc} 
\\[-1.8ex]\hline 
\\[-1.8ex] & \multicolumn{2}{c|}{\textbf{NIHSS Score}} & \multicolumn{2}{c}{\textbf{GCS Score}} \\ 
 & \textbf{HPR Coverage} & \textbf{Reader-Location} & \textbf{HPR Coverage} & \textbf{Reader-Location} \\ 
\hline \\[-1.8ex] 
 Age & $-$0.04 ($-$0.2, 0.1) & $-$0.1 ($-$0.2, 0.1) & 0.02 ($-$0.03, 0.1) & 0.02 ($-$0.03, 0.1) \\ 
  Sex: Male vs. Female & $-$0.7 ($-$3.8, 2.4) & $-$1.7 ($-$5.0, 1.7) & 0.03 ($-$1.1, 1.1) & 0.1 ($-$1.1, 1.3) \\ 
  TICHVol per 10 cc & 0.8$^{}$ ($-$0.003, 1.5) & 1.6$^{}$ (0.8, 2.4) & $-$0.2$^{}$ ($-$0.5, 0.02) & $-$0.5$^{}$ ($-$0.7, $-$0.2) \\ 
  HPR Coverage per 10\% & 2.0$^{}$ (1.1, 2.8) &  & $-$0.4$^{}$ ($-$0.6, $-$0.2) &  \\ 
  Reader-Based Location&&&& \\
\;\;Globus Pallidus &  & 4.5 ($-$2.9, 11.9) &  & $-$1.8 ($-$4.5, 0.8) \\ 
  \;\;Putamen &  & 4.2$^{}$ (0.3, 8.2) &  & $-$1.2$^{}$ ($-$2.6, 0.2) \\ 
  \;\;Thalamus &  & 4.8 ($-$4.1, 13.6) &  & $-$1.0 ($-$4.2, 2.1) \\ 
  Constant & 18.8$^{}$ (9.1, 28.4) & 19.6$^{}$ (7.5, 31.7) & 10.6$^{}$ (7.4, 13.7) & 11.2$^{}$ (6.9, 15.5) \\ 
\hline 
\end{tabular} 
  \caption{Regression Models for HPR-Based Analysis. The models for HPR coverage represent the best model based on the model-fit measures. We see that after adjusting for age, sex, and total baseline ICH volume, increasing 10\% coverage is expected to increase NIHSS score by 2.0 (95\% CI: 1.1, 2.8) points.  We see that all locations, compared to lobar hemorrhages have higher estimated NIHSS scores, but putaminal hemorrhages were significantly higher by 4.2 (95\% CI: 0.2, 8.2) points. Adjusting for other covariates, increasing 10\% coverage is expected to decrease GCS score by 0.4 (95\% CI: 0.6, 0.2) points.  We see that all locations, compared to lobar hemorrhages have lower estimated GCS scores, but none were statistically different.} 
  \label{f:beta} 
\end{table} 
