% latex table generated in R 3.0.2 by xtable 1.7-3 package
% Wed Jun 11 14:12:53 2014
\begin{table}[H]
\centering
\begin{tabular}{rr|cccc}
  \hline
{\bf Number of Voxels} & {\bf P-value} & {\bf Adjusted R$^2$} & {\bf R$^2$} & {\bf AIC} & {\bf RMSE} \\ 
  \hline
Location Model &  & 0.069 & 0.120 & 20.51 & 2.914 \\ 
  1000 & .0002 & 0.214 & 0.243 & 0.00 & 2.677 \\ 
  2000 & .0004 & 0.213 & 0.242 & 0.09 & 2.678 \\ 
  3000 & .0006 & 0.212 & 0.241 & 0.25 & 2.680 \\ 
  4669 & .0010 & 0.212 & 0.241 & 0.23 & 2.680 \\ 
  22858 & .0100 & 0.191 & 0.221 & 3.17 & 2.716 \\ 
  52368 & .0500 & 0.166 & 0.197 & 6.44 & 2.757 \\ 
   \hline
\end{tabular}
\caption{Table of model-fit measures for GCS score: reader-based location model vs. CT voxel-based HPR coverage models. We see that the model using a using the voxels with lowest 1000 p-values corresponds to the best model on all model-fit measures: adjusted and unadjusted $R^2$ are the highest; AIC, and RMSE are the smallest.  The models using HPR coverage have comparable fit measures, but each HPR coverage model outperforms the reader-based location model, notably in the $R^2$ measures.  Thus, we infer that using CT-based regions of interest may help in prediction of NIHSS score.} 
\label{t:gcs}
\end{table}
