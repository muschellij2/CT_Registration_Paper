\startmode[*mkii]
  \enableregime[utf-8]  
  \setupcolors[state=start]
\stopmode

% Enable hyperlinks
\setupinteraction[state=start, color=middleblue]

\setuppapersize [letter][letter]
\setuplayout    [width=middle,  backspace=1.5in, cutspace=1.5in,
                 height=middle, topspace=0.75in, bottomspace=0.75in]

\setuppagenumbering[location={footer,center}]

\setupbodyfont[11pt]

\setupwhitespace[medium]

\setuphead[chapter]      [style=\tfd]
\setuphead[section]      [style=\tfc]
\setuphead[subsection]   [style=\tfb]
\setuphead[subsubsection][style=\bf]

\setuphead[chapter, section, subsection, subsubsection][number=no]

\definedescription
  [description]
  [headstyle=bold, style=normal, location=hanging, width=broad, margin=1cm]

\setupitemize[autointro]    % prevent orphan list intro
\setupitemize[indentnext=no]

\setupthinrules[width=15em] % width of horizontal rules

\setupdelimitedtext
  [blockquote]
  [before={\blank[medium]},
   after={\blank[medium]},
   indentnext=no,
  ]


\starttext

{\bf Intracranial Hemorrhage Localization in a Population of Patients
using Registration-based Techniques in CT Imaging}\crlf
John Muschelli$^{1,\ast}$, Natalie L. Ullman$^{2}$, Elizabeth M.
Sweeney$^{3}$, Ani Eloyan$^{4}$, Neil Martin$^{5}$, Paul Vespa$^{6}$,
Daniel F. Hanley$^{5}$, Ciprian M. Crainiceanu$^{6}$\crlf
{\bf 1 John Muschelli Department of Biostatistics, Bloomberg School of
Public Health, Johns Hopkins University, Baltimore, MD, USA\crlf
{\bf 2 Natalie L. Ullman Department of Neurology, Division of Brain
Injury Outcomes, Johns Hopkins Medical Institutions, Baltimore, MD,
USA\crlf
{\bf 3 Elizabeth M. Sweeney, Department of Biostatistics, Bloomberg
School of Public Health, Johns Hopkins University, Baltimore, MD,
USA\crlf
{\bf 4 Ani Eloyan, Department of Biostatistics, Bloomberg School of
Public Health, Johns Hopkins University, Baltimore, MD, USA\crlf
{\bf 5 Neil Martin, Department of Neurosurgery, David Geffen School of
Medicine at UCLA, Los Angeles, California\crlf
{\bf 6 Paul Vespa, Department of Neurosurgery, David Geffen School of
Medicine at UCLA, Los Angeles, California\crlf
{\bf 7 Daniel F. Hanley Department of Neurology, Division of Brain
Injury Outcomes, Johns Hopkins Medical Institutions, Baltimore, MD,
USA\crlf
{\bf 8 Ciprian M. Crainiceanu Department of Biostatistics, Bloomberg
School of Public Health, Johns Hopkins University, Baltimore, MD,
USA\crlf
$\ast$ E-mail: Corresponding jmusche1@jhu.edu }}}}}}}}

\section{Abstract}

NA

{\bf Keywords:} Intracranial Hemorrhage; CT Imaging Analysis; 3D
Histograms;

\section{Introduction}

Intracranial hemorrhage (ICH) results from a blood vessel rupturing into
brain tissues and possibly the ventricles. Bleeding causes distension of
brain structures and increases the likelihood of intracranial pressure
(ICP) elevation. ICH accounts for 10-15\letterpercent{} of all strokes,
corresponding to approximately 80,000 annual cases , 30,000 deaths in
the US , and 5 million cases worldwide . CT scanning is widely available
and is the most commonly used diagnostic tool in patients with ICH .
Clinicians utilize CT to define the location of bleeding, clinically
assess severity, and plan patient management.

Despite robust correlation of cerebral location and functional
performance in normal humans, location of ICH surpisingly is not an
important factor in predicting severity of injury or prognosis .
Furthermore, recent clinical trials do not demonstrate an important role
for location as a factor associated with beneficial clinical outcome .

\subsection{Problems with Visual Inspection}

The classification of hemorrhage location is complicated for even the
best-trained neuroimage scientists. For example, a hemorrhage may extend
into multiple brain areas, distend tissues altering anatomic
relationships, and may break through the ventricular wall. Evaluating
these anatomic possibilities challenges even the best clinicians, thus
routine practice identifies a single location as the primary affected
anatomic region (e.g. caudate, putamen, etc.) or describes the location
of the edge of the hemorrhage in relation to a given landmark . Outcome
is strongly associated with hemorrhage volume; importantly, the
modulation of the relationship between volume and location has not been
studied. To investigate these anatomic issues, detailed localization
information can be obtained by registering scans to a common template to
provide refined anatomical localization information.

\subsection{Previous CT Registration Work}

Registration to template space is a crucial first step for any
across-patient analysis; this allows each patient's scan to be located
in the same stereotaxic space so information may be combined spatially
across scans. Moreover, brain atlases with spatially-defined anatomic
structures are available in template space. Recently, released the first
publicly available CT template of healthy adults in MNI (Montreal
Neurological Institute) space. We propose to utilize CT images from
patients enrolled in our clinical trials which have been acquired in a
standardized manner via the MISTIE (Minimally Invasive Surgery plus
recombinant-tissue plasminogen activator (rtPA) for Intracerebral
Evacuation) trial protocol and register them to template space. This
design provides a simple registration of the reference and template
images in a single imaging modality as opposed to more complex
referencing systems .

\subsection{Hypothesis}

We propose to use CT images from the MISTIE and ICES (Intraoperative
CT-Guided Endoscopic Surgery) trials to investigate the benefit or lack
thereof in utilizing anatomic location as a biologically plausible
predictor of ICH severity. Thus, we propose to test the hypothesis that
routine clinical anatomic localization was no different than
quantitative localization derived from registered-to-template images
with atlas-based labeling for prediction of severity of injury.

\section{Materials and Methods}

To address this hypothesis, we will 1) create a 3-dimensional (3D)
density map of hemorrhages occurring in a population of patients with
ICH; 2) provide detailed quantification of hemorrhage engagement of
individual neuroanatomic regions within the brain; 3) determine if
differences in location relate to two common disability scores: the
National Institutes of Health Stroke Scale (NIHSS) and the modified
Glasgow Coma Scale (GCS) ; and 4) generate a stroke region of engagement
that is likely to be associated with stroke severity as well as test its
predictive performance using within-sample validation.

All statistical analysis was done in the \type{R} statistical
programming language
(\useURL[url1][http://cran.r-project.org/][][\hyphenatedurl{http://cran.r-project.org/}]\from[url1]).

\subsection{Subjects and Demographics}

The population studied consists of 111 patients from MISTIE recruited
from 26 centers. For inclusion criteria, see . CT and clinical data were
collected as part of the Johns Hopkins Medicine IRB-approved MISTIE
research studies with written consent from participants.

The NIHSS and GCS scores were recorded at enrollment; $3$ patients did
not have a recorded NIHSS score, $1$ patient did not have a recorded GCS
score. These patients were excluded from the analyses associating NIHSS
and GCS scores and location, respectively. All patients were included in
the construction of the 3D histogram image. Descriptive demographic of
age, sex, race, baseline ICH and IVH volume, and NIHSS and GCS scores
are shown in Table~{[}t:dem{]}. The NIHSS score reflects stroke-related
impairment (higher is worse), while the GCS score reflects a patient's
level of consciousness (higher is better).

{[}Table~{[}t:dem{]} here.{]}

\subsection{Imaging Data}

The study protocol was executed with minor, but important, differences
across the 26 sites. Scans were acquired using GE ($N=46$), Siemens
($N=37$), Philips ($N=20$), and Toshiba ($N=8$) scanners. Gantry tilt
was observed in 87 scans. Slice thickness of the image varied within the
scan for 14 scans, referred to as variable slice thickness. For example,
a scan may have 10 millimeter (mm) slices at the top and bottom of the
brain, where no hematoma is present, but with 5mm slices in the middle
where the hematoma is seen (see Supplemental Figure~{[}f:reg{]}).
Therefore, the scans analyzed had different voxel (volume element)
dimensions and image resolution prior to registration to the template.
These conditions represent how scans are presented for evaluation in
many diagnostic cases.

\subsection{Hemorrhage Segmentation and Location Identification}

ICH was manually segmented on CT scans using the OsiriX imaging software
by expert readers (OsiriX v. 4.1, Pixmeo; Geneva, Switzerland). Readers
employed a semiautomated threshold-based approach using a Hounsfield
unit (HU) range of $40$ to $80$ to select potential regions of ICH ;
these regions were then further quality controlled and refined by
readers using direct inspection of images. Readers identified the
specific anatomic location most engaged by the ICH (Table~{[}t:dem{]}).

\subsection{Image Registration}

The DICOM (Digital Imaging and Communications in Medicine) data was
preprocessed to obtain a 3D brain image (see Supplemental
Section~{[}sec:processing{]} for details). The image was then spatially
registered to the CT template using the Clinical toolbox , which employs
the unified normalization-segmentation routine from the statistical
parametric mapping (version 8, SPM8, Wellcome Trust Centre for
Neuroimaging, London, United Kingdom) software in MATLAB (The Mathworks,
Natick, Massachusetts, USA). The binary hemorrhage mask was transformed
into the template space.

\subsection{Assessing Registration}

Determining the quality of registration was determined by visual
inspection by expert CT readers. Visual inspection is subjective, but
currently no automatic computer-based gold standard exists for assessing
registration results nor agreed-upon standards for assessing
registration quality in scans with large deformations of tissue as in
patients with ICH.

No scans were excluded due to inadequate registration.

\subsection{Histograms of ICH in the Brain}

To visualize and describe the localization of ICH, we first combined
information from registered masks of 111 patients. Using these ICH
masks, we obtained the $3$-dimensional (3D) histogram of ICH
localization for the study population. More precisely, for every voxel
in the template space, we calculated the proportion of patients who have
an ICH at that particular voxel. We used the \type{R} package
\type{brainR} to create an 3D interactive map of the 3D histogram . The
interactive map is located at
\useURL[url2][http://muschellij2.github.io/CT_Pipeline/index.html][][\hyphenatedurl{http://muschellij2.github.io/CT_Pipeline/index.html}]\from[url2].

\subsection{Prediction of Severity Score Based on Hemorrhage Location}

In the study population, 1045174 voxels had at least one patient with
ICH. We limited our analysis to voxels in the template space where at
least 10 patients exhibit ICH (166202 voxels) to optimize the models for
a substantial proportion of the population. We tested the association
between hemorrhage location and stroke severity as measured by the NIHSS
score and GCS score running a series of models, accounting for
confounders. At each voxel, we ran a linear regression model:

\startformula \label{eq:nihss_regression}
{\rm Y}_i=\beta_0+\beta_1(v) {\rm ICH}_i(v)+ \gamma X_i + \epsilon_{i}(v), \stopformula

where ${\rm Y}_i$ was either the NIHSS or GCS score for patient
$i=1,\ldots,111$, ${\rm ICH}_i(v)$ is a binary indicator where
${\rm ICH}_i(v) = 1$ if patient $i$'s ICH mask has a $1$ at voxel $v$,
and ${\rm ICH}_i(v) = 0$ otherwise. $X_i$ is a vector of
patient-specific confounders with effects $\gamma$ and $\epsilon_{i}(v)$
are assumed independent homoscedastic errors. The confounders, $X_i$,
either were excluded for \quotation{unadjusted} voxel-level model, or
contains a combination of $3$ patient-specific confounders: age, sex,
and total baseline ICH volume (TotalVol) in the \quotation{adjusted}
models. We used the unadjusted Wilcoxon rank-sum test on $Y$ at every
voxel to confirm that results are robust to the choice of test
statistic.

P-values of each voxel-wise model were calculated, testing the null
hypothesis $H_{0,v}:\beta_1(v)=0$, or in the case of the Wilcoxon
rank-sum test:
$H_{0,v}: Severity\{ICH(v) = 1\} = Severity\{ICH(v) = 0\}$ where
$Severity$ denotes the distribution of severity scores of patients.
Figure~{[}f:mods{]} displays the voxel-wise p-values from NIHSS score
models (see Supplemental Figure~{[}f:gcsmods{]} for GCS). The p-value
images displayed were not corrected for multiple comparisons since the
purpose is investigate regional brain anatomy where ICH engagement could
relate to severity score.

We did investigate whether individual locations are predictive of
severity score after accounting for multiple comparisons using a
Bonferonni correction with a family wise error rate of $\alpha=.05$. At
this stringent level no location was found to be significantly
associated with the scores.

\subsubsection{Highest Predictive Region Generation and Analysis}

Although no voxel passed this strict correction, voxels with low
p-values indicate candidate regions which may improve prediction of
severity scores. In order to create a patient-level covariate that
summarizes ICH location information, we created a sequence of nested
regions of interest by selecting voxels based on the smallest p-values
obtained from the undadjusted linear model, i.e. from a model where only
the voxel-level ICH indicator was used, but no other covariates. We call
these regions \quotation{highest predictive regions} (HPR) because they
contain the locations of those voxels that are most predictive of the
severity scores. We obtained $6$ different HPR, $3$ based on the
smallest $1000$, $2000$, or $3000$ lowest p-values and three based on
p-values thresholds of $.05$, $.01$, and $.001$. For each HPR, we
calculated the HPR “coverage" which represents the percentage of the
voxels in the HPR that were classified as hemorrhage in the
subject-specific image:
\startformula \text{Coverage}_i = \frac{\text{\# Voxels classified ICH in HPR for scan } i}{\text{\# Voxels in HPR}} \times 100\% \nonumber \stopformula
For example, if a patient's ICH covers the entire HPR, the coverage is
$100\%$, whereas if there is no overlap between the patient's ICH and
HPR then coverage is 0\letterpercent{}. This subject-specific covariate
is then used as a predictor of the severity score in the adjusted model:

\startformula {\rm Y}_i = \beta_0 + \beta_1 {\rm Coverage}_i + \gamma_1{\rm Age}_i  +\gamma_2{\rm Gender}_i +\gamma_3{\rm TotalVol}_i + \epsilon_{i} \label{eq:cov} \stopformula

We compared model~ to one using a categorical indicator of the
expert-specified ICH location, with categories: Thalamus ($N = 4$),
Globus Pallidus ($N = 6$), Putamen ($N = 68$), and Lobar ($N = 33$).
Prediction performance and model fit were assessed using $R^2$, adjusted
$R^2$, Akaike information criterion (AIC) , and root mean squared error
(RMSE).

\subsection{ICH Localization and Engagement}

{[}sec:engage{]} Although prediction of severity score is of interest,
standard practice conveys information based on known neuroanatomic
regions. We automatically calculated spatial ICH engagement by
neuroanatomic region using brain atlases with defined segmentations. We
used the “Eve" atlas , which segments gray matter (GM) and white matter
(WM) regions. Ventricular regions were not explicitly segmented; any
region not classified as GM or WM were classified as cerebrospinal fluid
(CSF).

From this atlas, we estimated for each patient scan: 1) the percent of
the ICH engaged by region and 2) the percent of each region engaged by
the ICH (see Supplemental Section~{[}sec:calc\low{p}erc{]} for further
details). These summaries of ICH engagement provide a much finer
description of location than what can currently be done by expert human
readers. For example, instead of classifying a region as a putaminal
bleed, we may indicate that an ICH engages 78\letterpercent{} of the
putamen. We did not directly compare reader-classified regions and the
most-engaged region classified by the atlas, as the atlas labels do not
directly map to the reader-classified categories.

We further summarized neuroanatomic engagement at the population level
(see Supplemental Section~{[}sec:calc\low{p}erc{]} for details).

\section{Results}

\subsection{Histograms of ICH in the Brain}

The non-spatial distribution of the prevalence of hemorrhages over all
voxels (Figure~{[}fig:StrokeHist{]}) shows the majority of voxels have a
low prevalence of ICH engagement; the median number of patients with ICH
at a given voxel is 3 (3\letterpercent{}), though a small group of
voxels ($V = 5685$) have a high prevalence of $> 40\%$ of the sample
population. Figure~{[}fig:StrokeHist{]} represents the 3D histogram of
hemorrhage prevalence, where colors represent the percentage of patients
with ICH engagement at that given area. This image indicates that ICH is
distributed medially in the brain in this cohort, with a lower
concentration at the cortical surface and higher on the left side of the
brain. Figure~{[}fig:StrokeHist{]} also indicates that the prevalence of
strokes in the extreme anterior and posterior areas of the brain are
very low. These observation may be a result of the inclusion criteria,
but the inclusion criteria did not specifically prefer some spatial
locations over others.

Overall, population maps such as these provide location information,
allow researchers to sharpen their hypotheses about ICH location, and
allow for such hypotheses to be tested. While this 3D histogram may not
generalize to other populations, it will be crucial to obtain such maps
for other studies and compare the degree of similarity to ours.

{[}Figure {[}fig:StrokeHist{]} here.{]}

Combining areas of engagement from the left and right sides of the brain
may be worthwhile, though combination of these areas may not be
straightforward for those by ICH that cross the mid-sagittal plane.

\subsection{Prediction of Functional Score Based on Hemorrhage Location}

To study the association between localization and stroke severity
scores, we have fit model~({[}eq:nihss\low{r}egression{]}) using a
sequence of potential confounding adjustments. More precisely, the five
voxel-wise linear models fit were:

rll $\mathcal{M}_1:$ & $ {\rm Y}_i =$ &

$\beta_0+\beta_1(v) {\rm ICH}_i(v) + \epsilon_{i}(v), $\crlf
$\mathcal{M}_2:$ & $ {\rm Y}_i = $ &

$ \beta_0+\beta_1(v) {\rm ICH}_i(v) + \gamma_1(v){\rm Age}_i + \epsilon_{i}(v), $\crlf
$\mathcal{M}_3:$ & $ {\rm Y}_i = $ &

$ \beta_0+\beta_1(v) {\rm ICH}_i(v) + \gamma_2(v){\rm Gender}_i + \epsilon_{i}(v), $\crlf
$\mathcal{M}_4:$ &$  {\rm Y}_i = $ &

$ \beta_0+\beta_1(v) {\rm ICH}_i(v) + \gamma_3(v){\rm TotalVol}_i + \epsilon_{i}(v),$\crlf
$\mathcal{M}_5:$ &$  {\rm Y}_i = $ &

$ \beta_0+\beta_1(v) {\rm ICH}_i(v) + \gamma_1(v){\rm Age}_i  +\gamma_2(v){\rm Gender}_i +\gamma_3(v){\rm TotalVol}_i + \epsilon_{i}(v),$\crlf

where $\mathcal{M}$ denotes a model. For consistency of notation (as it
is not an explicit model) we will refer to Wilcoxon rank-sum test as
$\mathcal{M}_6$.

No voxels were significantly related to NIHSS score or GCS score in any
model after using the Bonferroni correction. The p-values from the
models, with NIHSS score as the outcome, are presented in
Figure~{[}f:mods{]}, overlaid on a MRI T1 image for spatial localization
of structures ($\mathcal{M}_3$ and $\mathcal{M}_4$ are not shown, and
appear similar to Figure~{[}f:mods{]}).

The estimated association of location and NIHSS can easily be visualized
by comparing the relative location of high and low p-values.
Figure~{[}f:mods{]} indicates that the strongest associations are
clustered together in the immediate vicinity of the medial plane both
for the GCS (Supplemental Figure~{[}f:gcsmods{]}) and NIHSS scores. The
p-values are higher (more blue) in the models adjusted for age (not
shown), baseline ICH volume (not shown), and both age and baseline ICH
volume (Figure~{[}f:mods{]}) compared to the unadjusted model
$\mathcal{M}_1$.

{[}Figure {[}f:mods{]} here.{]}

\subsubsection{Highest Predictive Region Analysis}

Inspired by the inspection of Figure~{[}f:mods{]}, we investigated
whether we can define areas of the brain that can improve prediction of
stroke severity scores. To investigate this, have obtained 6 different
HPR, 3 based on the smallest $1000$, $2000$, and $3000$ p-values and
three based on all p-values below a particular threshold. We used three
thresholds: $.05$, $.01$, and $.001$, corresponding to HPR with $47736$,
$19047$, and $2422$ voxels, respectively, for NIHSS and $52368$,
$22858$, and $4669$ voxels for GCS, respectively. For illustration,
panel () in Figure~{[}f:roi{]} displays the region obtained by choosing
all the p-values smaller than $.01$ in the unadjusted NIHSS regression
on voxel location. Panel () in Figure~{[}f:roi{]} displays the region
obtained by retaining only the smallest $1000$ p-values for the
unadjusted regression of GCS score on ICH location. Although we show
three orthographic slices for each HPR, the entire regions are available
in MNI coordinates.

While the voxel-wise p-values identify areas that are potentially highly
associated with the outcome, we want to reduce the complex HPR to a
simple subject-specific covariate, HPR coverage. Panel~ of
Figure~{[}f:roi{]} displays the NIHSS score (y-axis) as a function of
the coverage of the HPR from Figure~{[}f:roi{]}. Similarly, Panel~ of
Figure~{[}f:roi{]} displays the GCS score as a function of the coverage
of the HPR in Figure~{[}f:roi{]} of Figure~{[}f:roi{]}. The blue line
represents a non-parametric LOESS fit to estimate the relationship
between severity and coverage, the red line represents an unadjusted
linear model fit. As expected, the larger the HPR coverage the higher
(more severe stroke) the NIHSS score and the lower (deeper
unconsciousness) the GCS score.

{[}Figure {[}f:roi{]} here.{]}

We further investigated how these regions perform compared to an ICH
location classification provided by expert readers. Table~{[}t:allres{]}
compares prediction performance based on the expert readers location
(labeled \quotation{Location model}) and the same models using HPR
coverage for NIHSS and GCS scores. Prediction performance is measured
primarily using the adjusted R$^2$, though conclusions were the same
using R$^2$, AIC, and RMSE. We conclude that all HPR coverage models
strongly outperform reader-classified location models. Indeed, for
NIHSS, the adjusted $R^2$ almost doubled from $0.129$ for the
reader-classified location model compared to $0.254$ for the best HPR
coverage model. For GCS, the adjusted $R^2$ more than tripled from
$0.069$ for the reader-classified location model to $0.214$ for the HPR
coverage model. The other HPR coverage models provide relatively similar
prediction performance. This indicates that the choice of one HPR versus
another may need to be based on other criteria, such as total HPR
volume, ICH population prevalence, and prior biological information. The
coefficients for the best HPR and location models can be seen in
Supplemental Table~{[}f:beta{]}.

{[}Table~{[}t:allres{]} here.{]}

\subsubsection{ICH Localization and Engagement}

Table~{[}t:breakdown{]} represents the 10 most-engaged regions for the
population 3D histogram as well as the HPR for the GCS and NIHSS score
analyses. The population ICH is engaged primarily in areas of the CSF,
such as the ventricles, the insular, and putaminal regions. The HPR
based on the NIHSS analysis engages primarily areas of the internal
capsule and ventricular regions. The HPR based on the GCS analysis
engages primarily the left thalamus and superior corona radiata.

{[}Table~{[}t:breakdown{]} here.{]}

Though Table~{[}t:breakdown{]} represents the empirical regions most
engaged, engagement by specified regions is commonly of interest. We
calculated the engagement of the thalamus, putamen and globus pallidus
by the population 3D histogram and the HPR for the GCS and NIHSS score
analyses (Supplemental Table~{[}t:area\low{b}reakdown{]}). The
population engagement represents the mean proportion of the population
with ICH engagement for that brain region. The HPR columns represent the
percent of voxels in that brain region that are in the HPR from NIHSS
and GCS scores. On average, 23\letterpercent{} of the putamen,
20\letterpercent{} of the globus pallidus, and 8\letterpercent{} of the
thalmus are engaged with ICH from patients in this study. The HPR from
the NIHSS analysis engages 40\letterpercent{} of the globus pallidus,
6\letterpercent{} of the putamen, and 9\letterpercent{} of the thalamus.
The HPR from the GCS analysis engages only 2\letterpercent{} of the
thalamus, but not the putamen nor the globus pallidus; the GCS HPR again
is only 1000 voxels. All engagement is higher on the left side compared
to the right.

\section{Discussion}

\subsection{Semi-Automated Localization of ICH}

We have characterized the localization of ICH in a population of stroke
patients by a 3D histogram of a population with ICH from prospective
clinical trials. We found, in this study, more patients had ICH
engagement on the left side of the brain and both left and right sided
ICHs are located towards the middle of the brain. We can use this
process to create a 3D histogram based on different studies or subgroups
of this population. As the values in the 3D histogram represent
proportions, these images can then be used to test differences between
the groups using standard proportion tests. This process allows us to
directly compare ICH location in 2 groups at a finer scale than
currently available.

Moreover, the pipeline described is semi-automated allowing for more
reproducible and objective analyses. The only non-automated steps in our
pipeline are the ICH segmentation and the export of ICH masks from
OsiriX.

\subsection{Voxel-wise Analysis of Severity Scores}

Voxel-wise hypothesis tests were performed using linear models and
Wilcoxon rank-sum tests for NIHSS and GCS scores. The resulting p-values
from these tests indicate that ICH at locations near the ventricles may
be related to severity scores.

None of the voxels passed the stringent Bonferonni correction due to the
large number of voxels being tested. As each voxel is not independent of
the other voxels as ICH is commonly a contiguous region of hemorrhage,
this correction is inappropriate. Other methods that attempt to account
of the smoothness properties, such as random field theory, may be more
powerful.

These maps allow us to explore the relationship of ICH location and
severity scores, which can be used for hypothesis generation and
testing. More importantly, these maps can be used as a potential
diagnostic clinical marker or guide surgical interventions. In addition,
we have shown the voxel-wise p-values provide a voxel screening
procedure to generate a HPR for patient-level analysis.

\subsection{Highest Predictive Region Analysis}

The HPR analysis rejects our hypothesis that routine clinical anatomic
localization was no different than quantitative localization derived
from registered-to-template images with atlas-based labeling for
prediction of severity of injury. Although prediction performance was
quite low even for the best models, prediction performance doubled or
tripled, depending on the severity score. These measures showed
demonstrable gains over using the reader-based categories commonly used
in analysis.

This analysis focused on NIHSS and GCS scores as these were available at
enrollment, but the process may be applied to long-term functional
scores. As these populations had groups with different interventions, we
aimed to analyze outcomes that were prior to any separation of the
groups induced by the intervention. This procedure can be used for any
patient outcome, such as the modified Rankin scale score at long-term
followup visits.

\subsection{Estimation of ICH Engagement with Neuroanatomic Regions}

Although 3D spatial maps are worthwhile for viewing, many clinicians
want description of location in terms of known neuroanatomic regions. We
demonstrate how a segmented atlas (Eve) can be used to automatically
describe ICH engagement by neuroanatomic regions at a patient or
population level. These measures are more interpretable for clinical
relevance and may translate to better determination of disability.
Additionally, these measures can be used to objectively compare
different groups or populations without the requiring expert readers to
determine ICH engagement.

\subsection{Conclusion}

The summaries of ICH engagement presented provide a much finer
description of location than currently done by expert human readers. We
also have shown that using image-based measurements can substantially
better predict severity scores compared to using location determined by
experts.

\subsection{Limitations}

Although this set of patients represent a large proportion
($79$\letterpercent{}) of the entire population ($N=141$) of the MISTIE
and ICES trials, the sample size is relatively small. The framework does
not impose a restriction on the number of patients to be processed; it
is extendable to much larger populations with thousands of patients.

The current description of hemorrhages and all subsequent analyses are
reliant on one specific registration technique. It is probably desirable
to try multiple registration approaches and compare results across
registrations.

One of the problems with the HPR is that they are obtained by using the
data twice: once to find the voxels that are most associated with the
outcome and the second time to obtain the region by selecting
(screening) the voxels with the smallest p-values. Cross-validation can
be used for internal validation and consists of splitting the data into
a training group used to generate the HPR and a testing group to
estimate model performance. Another approach is to externally validate
by studying the performance of the HPR on a subset of the MISTIE and
ICES patients not analyzed here. We will investigate external validation
in our future studies.

\section{Funding and Disclosures}

The project described was supported by the National Institutes of Health
(NIH) grant RO1EB012547 from the National Institute of Biomedical
Imaging And Bioengineering, training grant T32AG000247 from the National
Institute on Aging, NIH grants RO1NS060910 and RO1NS085211 from the
National Institute of Neurological Disorders and Stroke (NINDS), and by
NIH grant RO1MH095836 from the National Institute of Mental Health.

Dr. Daniel F. Hanley was awarded significant research support of grants
number R01NS046309 and 5U01NS062851 from NINDS. Johns Hopkins University
holds a use patent for intraventricular tissue plasminogen activator.

\section{Tables}

{[}ht{]}

\placetable[here]{none}
\starttable[|l|c|]
\HL
\NC {\bf Variable (N = 111)}
\NC {\bf N (\letterpercent{}) or Mean (SD)}
\NC\AR
\HL
\NC Age in Years: Mean (SD)
\NC 60.8 (11.2)
\NC\AR
\NC Gender: Female
\NC 35 (31.5\letterpercent{})
\NC\AR
\NC NIHSS Score: Mean (SD)
\NC 22.1 (8.7)
\NC\AR
\NC GCS Score: Mean (SD)
\NC 10.0 (3.0)
\NC\AR
\NC ICH Volume: Mean (SD)
\NC 37.4 (20.1)
\NC\AR
\NC IVH Volume: Mean (SD)
\NC 3.2 (6.3)
\NC\AR
\NC Race
\NC 
\NC\AR
\NC Caucasian not Hispanic
\NC 59 (53.2\letterpercent{})
\NC\AR
\NC African American not Hispanic
\NC 35 (31.5\letterpercent{})
\NC\AR
\NC Hispanic
\NC 12 (10.8\letterpercent{})
\NC\AR
\NC Asian or Pacific Islander
\NC 5 (4.5\letterpercent{})
\NC\AR
\NC Reader-Classified ICH Location
\NC 
\NC\AR
\NC Putamen
\NC 68 (61.3\letterpercent{})
\NC\AR
\NC Lobar
\NC 33 (29.7\letterpercent{})
\NC\AR
\NC Globus Pallidus
\NC 6 (5.4\letterpercent{})
\NC\AR
\NC Thalamus
\NC 4 (3.6\letterpercent{})
\NC\AR
\HL
\stoptable

{[}t:dem{]}

{[}H{]}

rr\letterbar{}cccc & {\bf P-value} & {\bf Adjusted R$^2$} & {\bf R$^2$}
& {\bf AIC} & {\bf RMSE}\crlf
\crlf
{\bf Location Model & {\bf & 0.129 & 0.178 & 18.60 & 8.116\crlf
{\bf 1000 & {\bf .0005 & 0.236 & 0.265 & 2.47 & 7.598\crlf
{\bf 2000 & {\bf .0009 & 0.234 & 0.263 & 2.81 & 7.610\crlf
{\bf 2422 & {\bf .0010 & 0.247 & 0.275 & 0.98 & 7.545\crlf
{\bf 3000 & {\bf .0013 & 0.244 & 0.272 & 1.46 & 7.562\crlf
{\bf 19047 & {\bf .0100 & 0.254 & 0.282 & 0.00 & 7.511\crlf
{\bf 47736 & {\bf .0500 & 0.248 & 0.276 & 0.77 & 7.538\crlf
\crlf
{\bf Location Model & {\bf & 0.069 & 0.120 & 20.51 & 2.914\crlf
{\bf 1000 & {\bf .0002 & 0.214 & 0.243 & 0.00 & 2.677\crlf
{\bf 2000 & {\bf .0004 & 0.213 & 0.242 & 0.09 & 2.678\crlf
{\bf 3000 & {\bf .0006 & 0.212 & 0.241 & 0.25 & 2.680\crlf
{\bf 4669 & {\bf .0010 & 0.212 & 0.241 & 0.23 & 2.680\crlf
{\bf 22858 & {\bf .0100 & 0.191 & 0.221 & 3.17 & 2.716\crlf
{\bf 52368 & {\bf .0500 & 0.166 & 0.197 & 6.44 & 2.757\crlf
}}}}}}}}}}}}}}}}}}}}}}}}}}}}

{[}t:allres{]}

{[}ht{]}

\placetable[here]{none}
\starttable[|l|c|c|c|]
\HL
\NC Area
\NC Population Prevalence
\NC NIHSS HPR
\NC GCS HPR
\NC\AR
\HL
\NC CSF
\NC 7.9
\NC 10.9
\NC 4.2
\NC\AR
\NC Insular
\NC 7.6
\NC 
\NC 
\NC\AR
\NC Superior temporal gyrus
\NC 5.5
\NC 
\NC 
\NC\AR
\NC Putamen
\NC 4.8
\NC 4.3
\NC 
\NC\AR
\NC External capsule
\NC 3.9
\NC 
\NC 
\NC\AR
\NC Superior corona radiata
\NC 3.7
\NC 11.0
\NC 27.9
\NC\AR
\NC Precentral gyrus
\NC 3.3
\NC 
\NC 
\NC\AR
\NC Precentral WM
\NC 3.1
\NC 
\NC 1.3
\NC\AR
\NC Superior temporal WM
\NC 3.1
\NC 
\NC 
\NC\AR
\NC Posterior limb of internal capsule
\NC 3.0
\NC 12.0
\NC 3.9
\NC\AR
\NC Thalamus
\NC 
\NC 10.1
\NC 33.9
\NC\AR
\NC Caudate nucleus
\NC 
\NC 8.4
\NC 9.6
\NC\AR
\NC Anterior limb of internal capsule
\NC 
\NC 6.8
\NC 
\NC\AR
\NC Globus pallidus
\NC 
\NC 6.0
\NC 
\NC\AR
\NC Superior longitudinal fasciculus
\NC 
\NC 4.5
\NC 5.9
\NC\AR
\NC Outside brain mask
\NC 
\NC 3.6
\NC 
\NC\AR
\NC Postcentral WM
\NC 
\NC 
\NC 6.7
\NC\AR
\NC Posterior corona radiata
\NC 
\NC 
\NC 3.1
\NC\AR
\NC Supramarginal WM
\NC 
\NC 
\NC 1.1
\NC\AR
\HL
\stoptable

{[}t:breakdown{]}

\section{Figure Legends}

{[}H{]}

{[}fig:StrokeHist{]}

{[}H{]}

{[}f:mods{]}

{[}H{]}

{[}f:roi{]}

\section{Supplemental Material}

\subsection{Image Processing: Brain Extraction, Reorientation,
Registration}

{[}sec:processing{]} To register the CT scan to the CT template, the
hemorrhage was excluded from the algorithm, i.e. \quotation{masked out},
using manual ICH segmentations. Binary image masks were created for the
hemorrhage ROI by setting voxel intensity to $1$ if the voxel was
classified as hemorrhage, and $0$ otherwise.

CT images were processed as follows:

\startitemize[n]
\item
  Export from OsiriX to DICOM format
\item
  Gantry tilt corrected (if applicable) using a customized MATLAB
  user-written script
  (\useURL[url3][http://bit.ly/1ltIM8c][][\hyphenatedurl{http://bit.ly/1ltIM8c}]\from[url3])
\item
  Converted to the Neuroimaging Informatics Technology Initiative
  (NIfTI) data format using \type{dcm2nii} (2009 version, provided with
  MRIcro )
\item
  The brain extraction tool (BET) , a function of the FSL neuroimaging
  software (v5.0.4) extracted a brain image.
\item
  The data was aligned to the anterior-posterior commissure line
  (\useURL[url4][http://bit.ly/1gUqMDw][][\hyphenatedurl{http://bit.ly/1gUqMDw}]\from[url4])
  from the brain image.
\item
  The Clinical toolbox was applied.
\stopitemize

An image with its brain-extracted counterpart can be seen in
Supplemental Figure~{[}f:reg{]}.

We thresholded the smoothed hemorrhage mask: let $S_i$ be the smoothed
image mask for person $i$, and $s_{ij}$ denotes voxel $j$ of that mask;
let $v_{ij}$ represents voxel $j$ for person $i$ of the thresholded
image. The was thresholded using the rule : \startformula v_{ij} =
\begin{cases}
1  & \text{if } s_{ij} \geq \frac{\min(S_i) + \max(S_i)}{2}\\
0  & \text{if } s_{ij} < \frac{\min(S_i) + \max(S_i)}{2}
\end{cases} \stopformula These binary ICH masks were used in analysis.

\subsection{Calculating Region Engagement from the Eve Atlas}

{[}sec:calc\low{p}erc{]} We also calculated the percent engagement of
regions for the best-performing HPR for the NIHSS and GCS score analyses
to characterize potential locations relating to these functional
outcomes. More explicitly, let $k$ denote the brain region (e.g.
Putamen) and let $\sum_{k} v_{k}$ represent the sum of the voxels for an
image (HPR or population ICH image) in that brain region. These $p_{k}$
represent the percent that brain region engages the ICH compared to
other regions (Table~{[}t:breakdown{]}):
\startformula p_{k} = \frac{\sum_{k} v_{k}}{\text{Sum of Image}} \stopformula
We have also calculated how much a brain region is engaged with the
population ICH or HPR images (Table~{[}t:area\low{b}reakdown{]}):
\startformula r_{k} = \frac{\sum_{k} v_{k}}{\# \text{Number of Voxels in Region}} \stopformula
These percentages ($r_{k}$) are at a region level rather than the $p_k$,
which are at an image level.

\subsection{Image Registration Results}

To illustrate registration results, we present Figure~{[}f:reg{]}:
manually segmented blood in the original space with the hemorrhage mask
in pink (panel~), this image after skull stripping (panel~), the
registered image and hemorrhage mask in template space (panel~), and the
ICH mask on the template (panel~). These images represent one patient
with variable slice thickness with a large hemorrhage. Though variable
slice thickness is present, the transformation morphs the image into the
full space of the template; therefore, non-linear registration seems to
reasonably account for variable slice thickness, most likely by
non-uniform scaling. We also see gross brain features remain relatively
unchanged, but large deformations of tissue, mainly due to ICH, appear
well preserved by registration.

{[}Figure~{[}f:reg{]} here.{]}

\subsection{Figures}

{[}H{]} {[}f:reg{]}

{[}H{]}

{[}f:gcsmods{]}

\subsection{Tables}

{[}!htbp{]}

@l@c@c\letterbar{}@c@c\crlf
\crlf
& &\crlf
& {\bf HPR Coverage} & {\bf Reader-Based} & {\bf HPR Coverage} &
{\bf Reader-Based}\crlf
\crlf
Age & $-$0.04 ($-$0.2, 0.1) & $-$0.1 ($-$0.2, 0.1) & 0.02 ($-$0.03, 0.1)
& 0.02 ($-$0.03, 0.1)\crlf
Sex: Male vs. Female & $-$0.7 ($-$3.8, 2.4) & $-$1.7 ($-$5.0, 1.7) &
0.03 ($-$1.1, 1.1) & 0.1 ($-$1.1, 1.3)\crlf
TICHVol per 10 cc & 0.8$^{}$ ($-$0.003, 1.5) & 1.6$^{}$ (0.8, 2.4) &
$-$0.2$^{}$ ($-$0.5, 0.02) & $-$0.5$^{}$ ($-$0.7, $-$0.2)\crlf
HPR Coverage per 10\letterpercent{} & 2.0$^{}$ (1.1, 2.8) & &
$-$0.4$^{}$ ($-$0.6, $-$0.2) &\crlf
Reader-Based Location&&&&\crlf
Globus Pallidus & & 4.5 ($-$2.9, 11.9) & & $-$1.8 ($-$4.5, 0.8)\crlf
Putamen & & 4.2$^{}$ (0.3, 8.2) & & $-$1.2$^{}$ ($-$2.6, 0.2)\crlf
Thalamus & & 4.8 ($-$4.1, 13.6) & & $-$1.0 ($-$4.2, 2.1)\crlf
Constant & 18.8$^{}$ (9.1, 28.4) & 19.6$^{}$ (7.5, 31.7) & 10.6$^{}$
(7.4, 13.7) & 11.2$^{}$ (6.9, 15.5)\crlf

{[}f:beta{]}

{[}ht{]}

\placetable[here]{none}
\starttable[|l|c|c|c|]
\HL
\NC Area
\NC Population Engagement
\NC NIHSS HPR
\NC GCS HPR
\NC\AR
\HL
\NC Globus Pallidus: Total
\NC 20.3
\NC 40.0
\NC 0.0
\NC\AR
\NC Globus Pallidus: Right
\NC 14.8
\NC 34.8
\NC 0.0
\NC\AR
\NC Globus Pallidus: Left
\NC 25.2
\NC 44.7
\NC 0.0
\NC\AR
\NC Putamen: Total
\NC 23.3
\NC 6.6
\NC 0.0
\NC\AR
\NC Putamen: Right
\NC 17.5
\NC 3.8
\NC 0.0
\NC\AR
\NC Putamen: Left
\NC 29.2
\NC 9.4
\NC 0.0
\NC\AR
\NC Thalamus: Total
\NC 7.9
\NC 8.9
\NC 1.7
\NC\AR
\NC Thalamus: Right
\NC 6.8
\NC 3.1
\NC 0.0
\NC\AR
\NC Thalamus: Left
\NC 9.1
\NC 14.6
\NC 3.4
\NC\AR
\HL
\stoptable

{[}t:area\low{b}reakdown{]}

\stoptext
